% \documentclass[fleqn]{jsarticle}

% \usepackage{graphicx}
% \usepackage{amsmath,amssymb}
% \usepackage{amsmath}
% \usepackage{fancyhdr}

\documentclass[dvipdfmx]{jsarticle}
\usepackage[dvipdfmx]{graphicx}
\usepackage{color}
\usepackage{amsmath}
\usepackage{amsfonts}
\usepackage{circuitikz}
\usepackage{cases}
\usepackage{otf}
\usepackage{ascmac,multicol,pifont,url}
\usepackage{booktabs}
\usepackage{bm}
\usepackage{listings,jlisting}
\usepackage{fancyhdr}



\pagestyle{fancy}
\fancyhead[RE,RO]{先端データ解析論 レポート}

\usepackage{xcolor}
\usepackage[justification=centering]{caption}
\usepackage{listings}
\renewcommand{\lstlistingname}{リスト}
\lstset{language=Python,%
        % basicstyle=\footnotesize,%
        basicstyle=\tiny,%
        commentstyle=\textit,%
        classoffset=1,%
        keywordstyle=\bfseries,%
      	frame=tRBl,framesep=5pt,%
      	showstringspaces=false,%
        linewidth=38em,
      	}%



% \DeclareMathOperator*{\argmin}{argmin}
\newcommand{\bmu}{{\bm{\mu}}}
\newcommand{\bmut}{\bm{\mu}^\top}
\newcommand{\tbmu}{\tilde{\bm{\mu}}}
\newcommand{\hbmu}{\hat{\bm{\mu}}}
\newcommand{\hbmut}{\hat{\bm{\mu}}^\top}
\newcommand{\tbmut}{\tilde{\bm{\mu}}^\top}
\newcommand{\sig}{\tilde{\bm{\Sigma}}}
\newcommand{\siginv}{\tilde{\bm{\Sigma}}^{-1}}
\newcommand{\bphi}{\bm{\phi}(\bm{x})}
\newcommand{\bphit}{\bm{\phi}(\bm{x})^\top}

\newcommand{\ex}{\mathbb{E}}
\newcommand{\bmx}{\bm{x}}
\newcommand{\bmxd}{\bm{x}{'}}
\newcommand{\ptrain}{p_{\text{train}}}
\newcommand{\hoge}{\mathbb{E}_{\bmxd \sim p_{\text{test}}, \bmx \sim q_\pi}}
\newcommand{\fuga}{\mathbb{E}_{\bmxd, \tilde{\bmx}' \sim p_{\text{test}}}}
\newcommand{\piyo}{\mathbb{E}_{\bmx, \tilde{\bmx} \sim q_\pi}}


\newcommand{\argmax}{\mathop{\rm argmax}\limits}
\newcommand{\argmin}{\mathop{\rm argmin}\limits}

\title{先端データ解析論 第8回レポート}
\author{電子情報学専攻 48-176403 石毛真修}


\begin{document}
\maketitle

\section*{1}

% 二乗ヒンジ損失に対する適応化正則化分類の式
% $$
% J(\bm{\mu}, \bm{\Sigma}) = \left(\max (0, 1-\bmut\bphi y) \right)^2 + \bphit \bm{\Sigma}\bphi + \gamma \left\{ \log \frac{\det(\sig)}{\det(\bm{\Sigma})} + (\siginv\bm{\Sigma}) + (\bmu-\tbmu)^\top \siginv (\bmu-\tbmu) \right\}
% $$
%
% に対して,$\bmu$で偏微分を行うと,
$\bmu$で偏微分すると,

% $$
% \frac{\partial J}{\partial \bmu} = \underbrace{\frac{\partial}{\partial \bmu} \left(\max (0, 1-\bmut\bphi y) \right)^2}_{(\#)} + \underbrace{\gamma \frac{\partial}{\partial \bmu} (\bmu-\tbmu)^\top \siginv (\bmu-\tbmu)}_{(\flat)}
% $$
$$
\frac{\partial J}{\partial \bmu} = \frac{\partial}{\partial \bmu} \left(\max (0, 1-\bmut\bphi y) \right)^2 + \gamma \frac{\partial}{\partial \bmu} (\bmu-\tbmu)^\top \siginv (\bmu-\tbmu)
$$

右辺の第1項は,$1-\bmut\bphi y > 0$のとき,
\begin{eqnarray*}
  \frac{\partial}{\partial \bmu} \left(\max (0, 1-\bmut\bphi y) \right)^2 &=& \frac{\partial}{\partial \bmu} \left(1-\bmut\bphi y \right)^2 \\
  &=& \frac{\partial}{\partial \bmu} \left(1 -2\bmut\bphi y + y^2\bmut\bphi\bmut\bphi \right) \\
  &=& \frac{\partial}{\partial \bmu} \left(1 -2y \bphit\bmu + y^2\bmut\bphi\bphit\bmu \right) \\
  &=& -2y\bphi + 2y^2\bphi\bphit\bmu \\
  &=& -2y\bphi\left(1 - \bmut\bphi y\right)
\end{eqnarray*}
$1-\bmut\bphi y < 0$のとき,
\begin{eqnarray*}
  \frac{\partial}{\partial \bmu} \left(\max (0, 1-\bmut\bphi y) \right)^2 = \bm{0}
\end{eqnarray*}
これらをまとめて,
$$
\frac{\partial}{\partial \bmu} \left(\max (0, 1-\bmut\bphi y) \right)^2= -2y\bphi\max\left(0, 1 - \bmut\bphi y\right)
$$


右辺の第2項は,
% $$
% (\flat) = \gamma \left(\siginv + (\siginv)^\top\right) (\bmu - \tbmu) = 2\gamma \siginv (\bmu - \tbmu)
% $$
$$
\gamma \frac{\partial}{\partial \bmu} (\bmu-\tbmu)^\top \siginv (\bmu-\tbmu) = \gamma \left(\siginv + (\siginv)^\top\right) (\bmu - \tbmu) = 2\gamma \siginv (\bmu - \tbmu)
$$



\newpage

第1項と第2項の式をまとめて,$\displaystyle\frac{\partial J}{\partial \bmu} = \bm{0}$と置くと,

$$
y\bphi\max\left(0, 1 - \hbmut\bphi y\right) = \gamma \siginv (\hbmu - \tbmu)
$$

が成り立つ.


これを$\hbmu$について解くと,$1-\hbmut\bphi y > 0$のときは

$$
\hbmu = \tbmu + \frac{y \left(1-\tbmut\bphi y\right)}{y^2 \bphit\sig\bphi + \gamma}\sig\bphi
$$

$1-\hbmut\bphi y < 0$ のときは

$$
\hbmu = \tbmu
$$

となる.以上より,解$\hbmu$は

$$
\hbmu = \tbmu + \frac{y \max\left(0, 1-\tbmut\bphi y\right)}{y^2 \bphit\sig\bphi + \gamma}\sig\bphi
$$

と表せる.


\end{document}


\end{document}
