\documentclass[fleqn]{jsarticle}

\usepackage{graphicx}
\usepackage{amsmath,amssymb}
\usepackage{amsmath}
\usepackage{fancyhdr}

\pagestyle{fancy}
\fancyhead[RE,RO]{先端データ解析論 レポート}

\usepackage{xcolor}
\usepackage[justification=centering]{caption}
\usepackage{listings}
\renewcommand{\lstlistingname}{リスト}
\lstset{language=Python,%
        % basicstyle=\footnotesize,%
        basicstyle=\tiny,%
        commentstyle=\textit,%
        classoffset=1,%
        keywordstyle=\bfseries,%
      	frame=tRBl,framesep=5pt,%
      	showstringspaces=false,%
        linewidth=38em,
      	}%

\begin{document}

\newcommand{\argmax}{\mathop{\rm argmax}\limits}
\newcommand{\argmin}{\mathop{\rm argmin}\limits}

\title{先端データ解析論 第5回レポート}
\author{電子情報学専攻 48-176403 石毛真修}
\maketitle



\section*{大問1.}
訓練標本入力の平均がゼロであり,2値の出力を $y \in \left\{ +\frac{n}{n_+}, -\frac{n}{n_-} \right\}$ であるとき,
線形モデル $f_\theta(\mathbf x) = \theta^{\mathrm T} \mathbf x$ を使った最小二乗分類器の決定境界の向きが,
\begin{eqnarray*}
  {\hat {\mathbf \Sigma}}^{-1} ({\hat {\mathbf \mu}}_+ - {\hat {\mathbf \mu}}_- )
\end{eqnarray*}
であることを示す.

\subsubsection*{証明}
最小二乗分類なので,
\begin{eqnarray*}
  \hat {\mathbf \theta} &=& argmin_\theta \frac{1}{2} \sum^n_{i=1} \left(f_\theta({\mathbf x}_i ) - y_i \right)\\
  &=& \left(X^{\mathrm T} X \right)^{-1} X^{\mathrm T} {\mathbf y}
\end{eqnarray*}

\begin{eqnarray*}
  X^{\mathrm T} {\mathbf y} &=& \sum^n_{i=1} {\mathbf x}_i y_i
  = \sum^n_{i:y=+} {\mathbf x}_i y_i + \sum^n_{i:y=-} {\mathbf x}_i y_i\\
  &=& \sum^n_{i:y=+} {\mathbf x}_i \frac{n}{n_+} + \sum^n_{i:y=-} {\mathbf x}_i \frac{- n}{n_-}
  = n (\hat{\mathbf \mu}_+ - \hat{\mathbf \mu}_-)
\end{eqnarray*}


一方,
\begin{eqnarray*}
  \hat {\mathbf \Sigma} = \frac{1}{n} X^{\mathrm T} X
\end{eqnarray*}
なので,
\begin{eqnarray*}
  \hat {\mathbf \theta} &=& \left(X^{\mathrm T} X \right)^{-1} X^{\mathrm T} {\mathbf y}
  = \left(n \hat {\mathbf \Sigma} \right)^{-1} X^{\mathrm T} {\mathbf y}\\
  &=& \left(n \hat {\mathbf \Sigma} \right)^{-1} n \left(\hat{\mathbf \mu}_+ - \hat{\mathbf \mu}_- \right) \\
  &=& {\hat {\mathbf \Sigma}}^{-1} \left(\hat{\mathbf \mu}_+ - \hat{\mathbf \mu}_- \right) \\
\end{eqnarray*}

決定境界は,訓練標本入力の平均がゼロであるので,
\begin{eqnarray*}
  {\mathbf \theta}^{\mathrm T} {\mathbf x} = \left\{{\hat {\mathbf \Sigma}}^{-1} (\hat{\mathbf \mu}_+ - \hat{\mathbf \mu}_-) \right\}^{\mathrm T} {\mathbf x} = 0
\end{eqnarray*}
となる.よって傾きは,${\hat {\mathbf \Sigma}}^{-1} ({\hat {\mathbf \mu}}_+ - {\hat {\mathbf \mu}}_- )$





\section*{大問2.}
ガウスカーネル回帰により,手書き数字を分類する.今回は,一対多による多クラス分類を行った.
まず,次のようなデータ直線$y=x$ に正規分布から生成されるノイズを載せたデータを作成する.

\begin{center}
\begin{tabular}{c}
  \begin{lstlisting}[]
    import numpy as np
    import pandas as pd
    import sklearn.units import shuffle


    # Read data
    train_data = [
        pd.read_csv('digit_train0.csv', header=None),
        pd.read_csv('digit_train1.csv', header=None),
        pd.read_csv('digit_train2.csv', header=None),
        pd.read_csv('digit_train3.csv', header=None),
        pd.read_csv('digit_train4.csv', header=None),
        pd.read_csv('digit_train5.csv', header=None),
        pd.read_csv('digit_train6.csv', header=None),
        pd.read_csv('digit_train7.csv', header=None),
        pd.read_csv('digit_train8.csv', header=None),
        pd.read_csv('digit_train9.csv', header=None),
    ]

    test_data = [
        pd.read_csv('digit_test0.csv', header=None),
        pd.read_csv('digit_test1.csv', header=None),
        pd.read_csv('digit_test2.csv', header=None),
        pd.read_csv('digit_test3.csv', header=None),
        pd.read_csv('digit_test4.csv', header=None),
        pd.read_csv('digit_test5.csv', header=None),
        pd.read_csv('digit_test6.csv', header=None),
        pd.read_csv('digit_test7.csv', header=None),
        pd.read_csv('digit_test8.csv', header=None),
        pd.read_csv('digit_test9.csv', header=None),
    ]


    def kern(x, c, h=0.2):
        norm = np.linalg.norm(x - c)
        return np.exp(- norm**2 / (2 * (h**2)))


    def kern_matrix(x_samples, h=0.2):
        def kerns(x, sample_X):
            return np.apply_along_axis(lambda xi: kern(x, xi), axis=1, arr=sample_X)
        return np.apply_along_axis(lambda x: kerns(x, x_samples), axis=1, arr=x_samples)


    def estimate_theta(samples_x, samples_y, lamb=0.1, h=0.2):
        K = kern_matrix(samples_x, h)
        Q = np.linalg.inv(np.matmul(K, K) + lamb * np.eye(len(samples_x)))
        p = np.matmul(np.transpose(K), samples_y)
        return np.matmul(Q, p)


    def kern_model_gen(x_samples, y_samples, lamb=0.1, h=0.2):
        est_theta = estimate_theta(x_samples, y_samples, lamb, h)
        def _model(x):
            return np.dot(est_theta, np.array([kern(x, xs, h) for xs in x_samples]))
        def v_model(X):
            return np.apply_along_axis(_model, axis=1, arr=X)
        return v_model


    Judgers = [class_judger_generator(i) for i in range(10)]


    def classifier(X):
        judges = np.array([j(X) for j in Judgers])
        return np.argmax(judges, axis=0)


    def accuracy():
        y = classifier(test_X)
        test_y = np.array([[i] * 200 for i in range(10)]).flatten()
        correct_num = np.sum(np.equal(y, test_y))
        return correct_num / len(test_y)


    print(accuracy())
  \end{lstlisting}
\end{tabular}{c}
\end{center}

このコードを実行した結果,96.3 \%の正解率であった.

\end{document}
